\documentclass{article}
\usepackage{geometry}
\usepackage{polyglossia}
\usepackage{fontspec}
\usepackage{flowfram}
\usepackage{tikz}
\usepackage[strict]{changepage}
\usepackage{lipsum}

\geometry{paperwidth=29.7cm,paperheight=7cm,margin=0cm}
\pagestyle{empty}

\newcommand{\pechafront}{
    \newdynamicframe{29.7cm}{7cm}{0cm}{0cm}[front]
    \setdynamiccontents*{front}{
        \vfil\hfil
        \begin{tikzpicture}
            \draw[color=black,very thick] (2.7,.93) rectangle +(24.3,5.14)
            (2.77,1) rectangle +(24.16,5)
            (3.77,1) rectangle +(.07,5)
            (25.97,1) rectangle +(.07,5);
            \node[rotate=-90] at (3.2,3.5) {\huge \thepage};
            \node[rotate=90] at (26.47,3.5) {\huge \thepage};
        \end{tikzpicture}
    }
    \newflowframe{21.7cm}{4.9cm}{4cm}{.9cm}[frontframe]
}

\newcommand{\pechaback}{
    \newdynamicframe{29.7cm}{7cm}{0cm}{0cm}[back]
    \setdynamiccontents*{back}{
        \begin{tikzpicture}
            \draw[color=black] (0,0) +(29.7,7)
            (3.77,1) rectangle +(22.16,5.12)
            (2.77,1) rectangle +(24.16,5.12);
            \node[rotate=90] at (3.43,3.5) {\huge \thepage};
            \node[rotate=-90] at (26.33,3.5) {\huge \thepage};
    \end{tikzpicture}}
    \newflowframe{21.7cm}{4.9cm}{4.0cm}{0.9cm}[backframe]
}

\sloppy

\setdefaultlanguage{tibetan}
\setmainfont{Tibetan Machine Uni}
\newfontfamily{\maintext}{Tibetan Machine Uni}
\strictpagecheck
\checkoddpage
\ifoddpage
    \pechafront
\else
    \pechaback
\fi
\begin{document}
% \lipsum
% \lipsum
\Large ༄༅། །ཕྱི་ལོ་ ༢༠༡༤ ཟླ་ ༦ ཚེས་ ༡༣ ཉིན་གྱི་སྔ་དྲོའི་ཆུ་ཚོད་ ༩།༣༠ ནས་ ༡༢།༣༠ ཙམ་བར་བོད་མིའི་སྒྲིག་འཛུགས་ཀྱི་ཕྱི་དྲིལ་དྲུང་ཆེ་བཀྲ་ཤིས་ཕུན་ཚོགས་ལགས་ཀྱིས་ཕྱི་དྲིལ་ལྷག་པ་ཚེ་རིང་དྲན་རྟེན་ཚོགས་ཁང་ནང་རྒྱ་གར་གྱི་རྒྱལ་ས་ལྡི་ལིར་རྟེན་གཞི་བྱས་པའི་༸གོང་ས་༸སྐྱབས་མགོན་ཏཱ་ལའི་བླ་མ་མཆོག་གི་ཀུན་ཕན་བདེ་རྩས་གོ་སྒྲིག་འོག་རྒྱ་གར་རྒྱལ་ཡོངས་མཐོ་སློབ་ཁག་ ༢༩ ནང་སློབ་གཉེར་བྱེད་བཞིན་པའི་སློབ་ཕྲུག་གྲངས་ ༣༣ དང་ཐུག་འཕྲད་ཀྱིས་བོད་མིའི་སྒྲིག་འཛུགས་ཀྱི་སྒྲོམ་གཞི་དང་སྲིད་བྱུས། དེ་བཞིན་ལྷན་ཁང་ཁག་གི་བྱེད་སྒོ་རྣམ་གསུམ་ངོ་སྤྲོད་དང་འབྲེལ་བོད་མིའི་སྒྲིག་འཛུགས་ངོས་ནས་དབུ་མའི་ལམ་གྱི་སྲིད་བྱུས་གཞིར་བཟུང་རྒྱ་ནག་གཞུང་དང་ལྷན་དུ་འབྲེལ་མོལ་བརྒྱུད་བོད་དོན་བདེན་མཐའ་གསལ་ཐབས་ལ་འབད་བརྩོན་ཞུ་བཞིན་ཡོད་སྐོར་སོགས་ཀྱི་གསུང་བཤད་ཟབ་རྒྱས་གནང་བ་མ་ཟད། དྲི་བར་ལན་འདེབས་ཀྱང་གནང་།སློབ་ཕྲུག་དེ་དག་ལ་ཟླ་བ་གཅིག་རིང་བཞུགས་སྒར་རྡ་སའི་ཁུལ་གྱི་བོད་པའི་དགོན་སྡེ་ཁག་ནང་བོད་ཀྱི་ཐུན་མིན་ཆོས་དང་རིག་གཞུང་སོགས་ཀྱི་སྐོར་ངོ་སྤྲོད་དང་སྦྱོང་བརྡར་སྤྲོད་རྒྱུ་ཡིན་འདུག སྤྱིར་སྒུ་རུ་ཀུལ་ (Gurukul Programme) ལས་འཆར་འདི་བཞིན་ཐོག་མར་ཕྱི་ལོ་ ༡༩༩༤ ལོར་འགོ་བཙུགས་གནང་བ་དང། དེ་སྔ་ལས་འཆར་འདིའི་འོག་མཐོ་སློབ་ཀྱི་སློབ་མ་དྲུག་བདུན་ཙམ་ལས་མཉམ་ཞུགས་གནང་མཁན་མེད་ཀྱང་། འདས་པའི་ལོ་དྲུག་གི་སྔོན་ཙམ་ནས་ལས་འཆར་དེར་མཉམ་ཞུགས་གནང་མཁན་མང་དུ་ཕྱིན་ནས་ལོ་ལྟར་སློབ་མ་བུ་བཅོ་ལྔ་དང་བུ་མོ་བཅོ་ལྔ་བདམས་ནས་རྡ་སར་རྟེན་གཞི་བྱས་པའི་བོད་པའི་ཆོས་དང་རིག་གཞུང་། གོམས་གཤིས་སོགས་ཉམས་མྱོང་གསོག་རྒྱུ་ཙམ་མ་ཟད། བོད་མིའི་སྒྲིག་འཛུགས་ཀྱི་ལས་ཁུངས་མ་ལག་དང་བཅས་པ་ཤེས་རྟོགས་བྱ་རྒྱུ་ཡིན་འདུག་ལོ་ ༢༠༡༤ ཟླ་ ༦ ཚེས་ ༡༣ ཉིན་གྱི་སྔ་དྲོའི་ཆུ་ཚོད་ ༩།༣༠ ནས་ ༡༢།༣༠ ཙམ་བར་བོད་མིའི་སྒྲིག་འཛུགས་ཀྱི་ཕྱི་དྲིལ་དྲུང་ཆེ་བཀྲ་ཤིས་ཕུན་ཚོགས་ལགས་ཀྱིས་ཕྱི་དྲིལ་ལྷག་པ་ཚེ་རིང་དྲན་རྟེན་ཚོགས་ཁང་ནང་རྒྱ་གར་གྱི་རྒྱལ་ས་ལྡི་ལིར་རྟེན་གཞི་བྱས་པའི་༸གོང་ས་༸སྐྱབས་མགོན་ཏཱ་ལའི་བླ་མ་མཆོག་གི་ཀུན་ཕན་བདེ་རྩས་གོ་སྒྲིག་འོག་རྒྱ་གར་རྒྱལ་ཡོངས་མཐོ་སློབ་ཁག་ ༢༩ ནང་སློབ་གཉེར་བྱེད་བཞིན་པའི་སློབ་ཕྲུག་གྲངས་ ༣༣ དང་ཐུག་འཕྲད་ཀྱིས་བོད་མིའི་སྒྲིག་འཛུགས་ཀྱི་སྒྲོམ་གཞི་དང་སྲིད་བྱུས། དེ་བཞིན་ལྷན་ཁང་ཁག་གི་བྱེད་སྒོ་རྣམ་གསུམ་ངོ་སྤྲོད་དང་འབྲེལ་བོད་མིའི་སྒྲིག་འཛུགས་ངོས་ནས་དབུ་མའི་ལམ་གྱི་སྲིད་བྱུས་གཞིར་བཟུང་རྒྱ་ནག་གཞུང་དང་ལྷན་དུ་འབྲེལ་མོལ་བརྒྱུད་བོད་དོན་བདེན་མཐའ་གསལ་ཐབས་ལ་འབད་བརྩོན་ཞུ་བཞིན་ཡོད་སྐོར་སོགས་ཀྱི་གསུང་བཤད་ཟབ་རྒྱས་གནང་བ་མ་ཟད། དྲི་བར་ལན་འདེབས་ཀྱང་གནང་།སློབ་ཕྲུག་དེ་དག་ལ་ཟླ་བ་གཅིག་རིང་བཞུགས་སྒར་རྡ་སའི་ཁུལ་གྱི་བོད་པའི་དགོན་སྡེ་ཁག་ནང་བོད་ཀྱི་ཐུན་མིན་ཆོས་དང་རིག་གཞུང་སོགས་ཀྱི་སྐོར་ངོ་སྤྲོད་དང་སྦྱོང་བརྡར་སྤྲོད་རྒྱུ་ཡིན་འདུག སྤྱིར་སྒུ་རུ་ཀུལ་ (Gurukul Programme) ལས་འཆར་འདི་བཞིན་ཐོག་མར་ཕྱི་ལོ་ ༡༩༩༤ ལོར་འགོ་བཙུགས་གནང་བ་དང། དེ་སྔ་ལས་འཆར་འདིའི་འོག་མཐོ་སློབ་ཀྱི་སློབ་མ་དྲུག་བདུན་ཙམ་ལས་མཉམ་ཞུགས་གནང་མཁན་མེད་ཀྱང་། འདས་པའི་ལོ་དྲུག་གི་སྔོན་ཙམ་ནས་ལས་འཆར་དེར་མཉམ་ཞུགས་གནང་མཁན་མང་དུ་ཕྱིན་ནས་ལོ་ལྟར་སློབ་མ་བུ་བཅོ་ལྔ་དང་བུ་མོ་བཅོ་ལྔ་བདམས་ནས་རྡ་སར་རྟེན་གཞི་བྱས་པའི་བོད་པའི་ཆོས་དང་རིག་གཞུང་། གོམས་གཤིས་སོགས་ཉམས་མྱོང་གསོག་རྒྱུ་ཙམ་མ་ཟད། བོད་མིའི་སྒྲིག་འཛུགས་ཀྱི་ལས་ཁུངས་མ་ལག་དང་བཅས་པ་ཤེས་རྟོགས་བྱ་རྒྱུ་ཡིན་འདུག་ལོ་ ༢༠༡༤ ཟླ་ ༦ ཚེས་ ༡༣ ཉིན་གྱི་སྔ་དྲོའི་ཆུ་ཚོད་ ༩།༣༠ ནས་ ༡༢།༣༠ ཙམ་བར་བོད་མིའི་སྒྲིག་འཛུགས་ཀྱི་ཕྱི་དྲིལ་དྲུང་ཆེ་བཀྲ་ཤིས་ཕུན་ཚོགས་ལགས་ཀྱིས་ཕྱི་དྲིལ་ལྷག་པ་ཚེ་རིང་དྲན་རྟེན་ཚོགས་ཁང་ནང་རྒྱ་གར་གྱི་རྒྱལ་ས་ལྡི་ལིར་རྟེན་གཞི་བྱས་པའི་༸གོང་ས་༸སྐྱབས་མགོན་ཏཱ་ལའི་བླ་མ་མཆོག་གི་ཀུན་ཕན་བདེ་རྩས་གོ་སྒྲིག་འོག་རྒྱ་གར་རྒྱལ་ཡོངས་མཐོ་སློབ་ཁག་ ༢༩ ནང་སློབ་གཉེར་བྱེད་བཞིན་པའི་སློབ་ཕྲུག་གྲངས་ ༣༣ དང་ཐུག་འཕྲད་ཀྱིས་བོད་མིའི་སྒྲིག་འཛུགས་ཀྱི་སྒྲོམ་གཞི་དང་སྲིད་བྱུས། དེ་བཞིན་ལྷན་ཁང་ཁག་གི་བྱེད་སྒོ་རྣམ་གསུམ་ངོ་སྤྲོད་དང་འབྲེལ་བོད་མིའི་སྒྲིག་འཛུགས་ངོས་ནས་དབུ་མའི་ལམ་གྱི་སྲིད་བྱུས་གཞིར་བཟུང་རྒྱ་ནག་གཞུང་དང་ལྷན་དུ་འབྲེལ་མོལ་བརྒྱུད་བོད་དོན་བདེན་མཐའ་གསལ་ཐབས་ལ་འབད་བརྩོན་ཞུ་བཞིན་ཡོད་སྐོར་སོགས་ཀྱི་གསུང་བཤད་ཟབ་རྒྱས་གནང་བ་མ་ཟད། དྲི་བར་ལན་འདེབས་ཀྱང་གནང་།སློབ་ཕྲུག་དེ་དག་ལ་ཟླ་བ་གཅིག་རིང་བཞུགས་སྒར་རྡ་སའི་ཁུལ་གྱི་བོད་པའི་དགོན་སྡེ་ཁག་ནང་བོད་ཀྱི་ཐུན་མིན་ཆོས་དང་རིག་གཞུང་སོགས་ཀྱི་སྐོར་ངོ་སྤྲོད་དང་སྦྱོང་བརྡར་སྤྲོད་རྒྱུ་ཡིན་འདུག སྤྱིར་སྒུ་རུ་ཀུལ་ (Gurukul Programme) ལས་འཆར་འདི་བཞིན་ཐོག་མར་ཕྱི་ལོ་ ༡༩༩༤ ལོར་འགོ་བཙུགས་གནང་བ་དང། དེ་སྔ་ལས་འཆར་འདིའི་འོག་མཐོ་སློབ་ཀྱི་སློབ་མ་དྲུག་བདུན་ཙམ་ལས་མཉམ་ཞུགས་གནང་མཁན་མེད་ཀྱང་། འདས་པའི་ལོ་དྲུག་གི་སྔོན་ཙམ་ནས་ལས་འཆར་དེར་མཉམ་ཞུགས་གནང་མཁན་མང་དུ་ཕྱིན་ནས་ལོ་ལྟར་སློབ་མ་བུ་བཅོ་ལྔ་དང་བུ་མོ་བཅོ་ལྔ་བདམས་ནས་རྡ་སར་རྟེན་གཞི་བྱས་པའི་བོད་པའི་ཆོས་དང་རིག་གཞུང་། གོམས་གཤིས་སོགས་ཉམས་མྱོང་གསོག་རྒྱུ་ཙམ་མ་ཟད། བོད་མིའི་སྒྲིག་འཛུགས་ཀྱི་ལས་ཁུངས་མ་ལག་དང་བཅས་པ་ཤེས་རྟོགས་བྱ་རྒྱུ་ཡིན་འདུག་ལོ་ ༢༠༡༤ ཟླ་ ༦ ཚེས་ ༡༣ ཉིན་གྱི་སྔ་དྲོའི་ཆུ་ཚོད་ ༩།༣༠ ནས་ ༡༢།༣༠ ཙམ་བར་བོད་མིའི་སྒྲིག་འཛུགས་ཀྱི་ཕྱི་དྲིལ་དྲུང་ཆེ་བཀྲ་ཤིས་ཕུན་ཚོགས་ལགས་ཀྱིས་ཕྱི་དྲིལ་ལྷག་པ་ཚེ་རིང་དྲན་རྟེན་ཚོགས་ཁང་ནང་རྒྱ་གར་གྱི་རྒྱལ་ས་ལྡི་ལིར་རྟེན་གཞི་བྱས་པའི་༸གོང་ས་༸སྐྱབས་མགོན་ཏཱ་ལའི་བླ་མ་མཆོག་གི་ཀུན་ཕན་བདེ་རྩས་གོ་སྒྲིག་འོག་རྒྱ་གར་རྒྱལ་ཡོངས་མཐོ་སློབ་ཁག་ ༢༩ ནང་སློབ་གཉེར་བྱེད་བཞིན་པའི་སློབ་ཕྲུག་གྲངས་ ༣༣ དང་ཐུག་འཕྲད་ཀྱིས་བོད་མིའི་སྒྲིག་འཛུགས་ཀྱི་སྒྲོམ་གཞི་དང་སྲིད་བྱུས། དེ་བཞིན་ལྷན་ཁང་ཁག་གི་བྱེད་སྒོ་རྣམ་གསུམ་ངོ་སྤྲོད་དང་འབྲེལ་བོད་མིའི་སྒྲིག་འཛུགས་ངོས་ནས་དབུ་མའི་ལམ་གྱི་སྲིད་བྱུས་གཞིར་བཟུང་རྒྱ་ནག་གཞུང་དང་ལྷན་དུ་འབྲེལ་མོལ་བརྒྱུད་བོད་དོན་བདེན་མཐའ་གསལ་ཐབས་ལ་འབད་བརྩོན་ཞུ་བཞིན་ཡོད་སྐོར་སོགས་ཀྱི་གསུང་བཤད་ཟབ་རྒྱས་གནང་བ་མ་ཟད། དྲི་བར་ལན་འདེབས་ཀྱང་གནང་།སློབ་ཕྲུག་དེ་དག་ལ་ཟླ་བ་གཅིག་རིང་བཞུགས་སྒར་རྡ་སའི་ཁུལ་གྱི་བོད་པའི་དགོན་སྡེ་ཁག་ནང་བོད་ཀྱི་ཐུན་མིན་ཆོས་དང་རིག་གཞུང་སོགས་ཀྱི་སྐོར་ངོ་སྤྲོད་དང་སྦྱོང་བརྡར་སྤྲོད་རྒྱུ་ཡིན་འདུག སྤྱིར་སྒུ་རུ་ཀུལ་ (Gurukul Programme) ལས་འཆར་འདི་བཞིན་ཐོག་མར་ཕྱི་ལོ་ ༡༩༩༤ ལོར་འགོ་བཙུགས་གནང་བ་དང། དེ་སྔ་ལས་འཆར་འདིའི་འོག་མཐོ་སློབ་ཀྱི་སློབ་མ་དྲུག་བདུན་ཙམ་ལས་མཉམ་ཞུགས་གནང་མཁན་མེད་ཀྱང་། འདས་པའི་ལོ་དྲུག་གི་སྔོན་ཙམ་ནས་ལས་འཆར་དེར་མཉམ་ཞུགས་གནང་མཁན་མང་དུ་ཕྱིན་ནས་ལོ་ལྟར་སློབ་མ་བུ་བཅོ་ལྔ་དང་བུ་མོ་བཅོ་ལྔ་བདམས་ནས་རྡ་སར་རྟེན་གཞི་བྱས་པའི་བོད་པའི་ཆོས་དང་རིག་གཞུང་། གོམས་གཤིས་སོགས་ཉམས་མྱོང་གསོག་རྒྱུ་ཙམ་མ་ཟད། བོད་མིའི་སྒྲིག་འཛུགས་ཀྱི་ལས་ཁུངས་མ་ལག་དང་བཅས་པ་ཤེས་རྟོགས་བྱ་རྒྱུ་ཡིན་འདུག་ལོ་ ༢༠༡༤ ཟླ་ ༦ ཚེས་ ༡༣ ཉིན་གྱི་སྔ་དྲོའི་ཆུ་ཚོད་ ༩།༣༠ ནས་ ༡༢།༣༠ ཙམ་བར་བོད་མིའི་སྒྲིག་འཛུགས་ཀྱི་ཕྱི་དྲིལ་དྲུང་ཆེ་བཀྲ་ཤིས་ཕུན་ཚོགས་ལགས་ཀྱིས་ཕྱི་དྲིལ་ལྷག་པ་ཚེ་རིང་དྲན་རྟེན་ཚོགས་ཁང་ནང་རྒྱ་གར་གྱི་རྒྱལ་ས་ལྡི་ལིར་རྟེན་གཞི་བྱས་པའི་༸གོང་ས་༸སྐྱབས་མགོན་ཏཱ་ལའི་བླ་མ་མཆོག་གི་ཀུན་ཕན་བདེ་རྩས་གོ་སྒྲིག་འོག་རྒྱ་གར་རྒྱལ་ཡོངས་མཐོ་སློབ་ཁག་ ༢༩ ནང་སློབ་གཉེར་བྱེད་བཞིན་པའི་སློབ་ཕྲུག་གྲངས་ ༣༣ དང་ཐུག་འཕྲད་ཀྱིས་བོད་མིའི་སྒྲིག་འཛུགས་ཀྱི་སྒྲོམ་གཞི་དང་སྲིད་བྱུས། དེ་བཞིན་ལྷན་ཁང་ཁག་གི་བྱེད་སྒོ་རྣམ་གསུམ་ངོ་སྤྲོད་དང་འབྲེལ་བོད་མིའི་སྒྲིག་འཛུགས་ངོས་ནས་དབུ་མའི་ལམ་གྱི་སྲིད་བྱུས་གཞིར་བཟུང་རྒྱ་ནག་གཞུང་དང་ལྷན་དུ་འབྲེལ་མོལ་བརྒྱུད་བོད་དོན་བདེན་མཐའ་གསལ་ཐབས་ལ་འབད་བརྩོན་ཞུ་བཞིན་ཡོད་སྐོར་སོགས་ཀྱི་གསུང་བཤད་ཟབ་རྒྱས་གནང་བ་མ་ཟད། དྲི་བར་ལན་འདེབས་ཀྱང་གནང་།སློབ་ཕྲུག་དེ་དག་ལ་ཟླ་བ་གཅིག་རིང་བཞུགས་སྒར་རྡ་སའི་ཁུལ་གྱི་བོད་པའི་དགོན་སྡེ་ཁག་ནང་བོད་ཀྱི་ཐུན་མིན་ཆོས་དང་རིག་གཞུང་སོགས་ཀྱི་སྐོར་ངོ་སྤྲོད་དང་སྦྱོང་བརྡར་སྤྲོད་རྒྱུ་ཡིན་འདུག སྤྱིར་སྒུ་རུ་ཀུལ་ (Gurukul Programme) ལས་འཆར་འདི་བཞིན་ཐོག་མར་ཕྱི་ལོ་ ༡༩༩༤ ལོར་འགོ་བཙུགས་གནང་བ་དང། དེ་སྔ་ལས་འཆར་འདིའི་འོག་མཐོ་སློབ་ཀྱི་སློབ་མ་དྲུག་བདུན་ཙམ་ལས་མཉམ་ཞུགས་གནང་མཁན་མེད་ཀྱང་། འདས་པའི་ལོ་དྲུག་གི་སྔོན་ཙམ་ནས་ལས་འཆར་དེར་མཉམ་ཞུགས་གནང་མཁན་མང་དུ་ཕྱིན་ནས་ལོ་ལྟར་སློབ་མ་བུ་བཅོ་ལྔ་དང་བུ་མོ་བཅོ་ལྔ་བདམས་ནས་རྡ་སར་རྟེན་གཞི་བྱས་པའི་བོད་པའི་ཆོས་དང་རིག་གཞུང་། གོམས་གཤིས་སོགས་ཉམས་མྱོང་གསོག་རྒྱུ་ཙམ་མ་ཟད། བོད་མིའི་སྒྲིག་འཛུགས་ཀྱི་ལས་ཁུངས་མ་ལག་དང་བཅས་པ་ཤེས་རྟོགས་བྱ་རྒྱུ་ཡིན་འདུག་ལོ་ ༢༠༡༤ ཟླ་ ༦ ཚེས་ ༡༣ ཉིན་གྱི་སྔ་དྲོའི་ཆུ་ཚོད་ ༩།༣༠ ནས་ ༡༢།༣༠ ཙམ་བར་བོད་མིའི་སྒྲིག་འཛུགས་ཀྱི་ཕྱི་དྲིལ་དྲུང་ཆེ་བཀྲ་ཤིས་ཕུན་ཚོགས་ལགས་ཀྱིས་ཕྱི་དྲིལ་ལྷག་པ་ཚེ་རིང་དྲན་རྟེན་ཚོགས་ཁང་ནང་རྒྱ་གར་གྱི་རྒྱལ་ས་ལྡི་ལིར་རྟེན་གཞི་བྱས་པའི་༸གོང་ས་༸སྐྱབས་མགོན་ཏཱ་ལའི་བླ་མ་མཆོག་གི་ཀུན་ཕན་བདེ་རྩས་གོ་སྒྲིག་འོག་རྒྱ་གར་རྒྱལ་ཡོངས་མཐོ་སློབ་ཁག་ ༢༩ ནང་སློབ་གཉེར་བྱེད་བཞིན་པའི་སློབ་ཕྲུག་གྲངས་ ༣༣ དང་ཐུག་འཕྲད་ཀྱིས་བོད་མིའི་སྒྲིག་འཛུགས་ཀྱི་སྒྲོམ་གཞི་དང་སྲིད་བྱུས། དེ་བཞིན་ལྷན་ཁང་ཁག་གི་བྱེད་སྒོ་རྣམ་གསུམ་ངོ་སྤྲོད་དང་འབྲེལ་བོད་མིའི་སྒྲིག་འཛུགས་ངོས་ནས་དབུ་མའི་ལམ་གྱི་སྲིད་བྱུས་གཞིར་བཟུང་རྒྱ་ནག་གཞུང་དང་ལྷན་དུ་འབྲེལ་མོལ་བརྒྱུད་བོད་དོན་བདེན་མཐའ་གསལ་ཐབས་ལ་འབད་བརྩོན་ཞུ་བཞིན་ཡོད་སྐོར་སོགས་ཀྱི་གསུང་བཤད་ཟབ་རྒྱས་གནང་བ་མ་ཟད། དྲི་བར་ལན་འདེབས་ཀྱང་གནང་།སློབ་ཕྲུག་དེ་དག་ལ་ཟླ་བ་གཅིག་རིང་བཞུགས་སྒར་རྡ་སའི་ཁུལ་གྱི་བོད་པའི་དགོན་སྡེ་ཁག་ནང་བོད་ཀྱི་ཐུན་མིན་ཆོས་དང་རིག་གཞུང་སོགས་ཀྱི་སྐོར་ངོ་སྤྲོད་དང་སྦྱོང་བརྡར་སྤྲོད་རྒྱུ་ཡིན་འདུག སྤྱིར་སྒུ་རུ་ཀུལ་ (Gurukul Programme) ལས་འཆར་འདི་བཞིན་ཐོག་མར་ཕྱི་ལོ་ ༡༩༩༤ ལོར་འགོ་བཙུགས་གནང་བ་དང། དེ་སྔ་ལས་འཆར་འདིའི་འོག་མཐོ་སློབ་ཀྱི་སློབ་མ་དྲུག་བདུན་ཙམ་ལས་མཉམ་ཞུགས་གནང་མཁན་མེད་ཀྱང་། འདས་པའི་ལོ་དྲུག་གི་སྔོན་ཙམ་ནས་ལས་འཆར་དེར་མཉམ་ཞུགས་གནང་མཁན་མང་དུ་ཕྱིན་ནས་ལོ་ལྟར་སློབ་མ་བུ་བཅོ་ལྔ་དང་བུ་མོ་བཅོ་ལྔ་བདམས་ནས་རྡ་སར་རྟེན་གཞི་བྱས་པའི་བོད་པའི་ཆོས་དང་རིག་གཞུང་། གོམས་གཤིས་སོགས་ཉམས་མྱོང་གསོག་རྒྱུ་ཙམ་མ་ཟད། བོད་མིའི་སྒྲིག་འཛུགས་ཀྱི་ལས་ཁུངས་མ་ལག་དང་བཅས་པ་ཤེས་རྟོགས་བྱ་རྒྱུ་ཡིན་འདུག་ལོ་ ༢༠༡༤ ཟླ་ ༦ ཚེས་ ༡༣ ཉིན་གྱི་སྔ་དྲོའི་ཆུ་ཚོད་ ༩།༣༠ ནས་ ༡༢།༣༠ ཙམ་བར་བོད་མིའི་སྒྲིག་འཛུགས་ཀྱི་ཕྱི་དྲིལ་དྲུང་ཆེ་བཀྲ་ཤིས་ཕུན་ཚོགས་ལགས་ཀྱིས་ཕྱི་དྲིལ་ལྷག་པ་ཚེ་རིང་དྲན་རྟེན་ཚོགས་ཁང་ནང་རྒྱ་གར་གྱི་རྒྱལ་ས་ལྡི་ལིར་རྟེན་གཞི་བྱས་པའི་༸གོང་ས་༸སྐྱབས་མགོན་ཏཱ་ལའི་བླ་མ་མཆོག་གི་ཀུན་ཕན་བདེ་རྩས་གོ་སྒྲིག་འོག་རྒྱ་གར་རྒྱལ་ཡོངས་མཐོ་སློབ་ཁག་ ༢༩ ནང་སློབ་གཉེར་བྱེད་བཞིན་པའི་སློབ་ཕྲུག་གྲངས་ ༣༣ དང་ཐུག་འཕྲད་ཀྱིས་བོད་མིའི་སྒྲིག་འཛུགས་ཀྱི་སྒྲོམ་གཞི་དང་སྲིད་བྱུས། དེ་བཞིན་ལྷན་ཁང་ཁག་གི་བྱེད་སྒོ་རྣམ་གསུམ་ངོ་སྤྲོད་དང་འབྲེལ་བོད་མིའི་སྒྲིག་འཛུགས་ངོས་ནས་དབུ་མའི་ལམ་གྱི་སྲིད་བྱུས་གཞིར་བཟུང་རྒྱ་ནག་གཞུང་དང་ལྷན་དུ་འབྲེལ་མོལ་བརྒྱུད་བོད་དོན་བདེན་མཐའ་གསལ་ཐབས་ལ་འབད་བརྩོན་ཞུ་བཞིན་ཡོད་སྐོར་སོགས་ཀྱི་གསུང་བཤད་ཟབ་རྒྱས་གནང་བ་མ་ཟད། དྲི་བར་ལན་འདེབས་ཀྱང་གནང་།སློབ་ཕྲུག་དེ་དག་ལ་ཟླ་བ་གཅིག་རིང་བཞུགས་སྒར་རྡ་སའི་ཁུལ་གྱི་བོད་པའི་དགོན་སྡེ་ཁག་ནང་བོད་ཀྱི་ཐུན་མིན་ཆོས་དང་རིག་གཞུང་སོགས་ཀྱི་སྐོར་ངོ་སྤྲོད་དང་སྦྱོང་བརྡར་སྤྲོད་རྒྱུ་ཡིན་འདུག སྤྱིར་སྒུ་རུ་ཀུལ་ (Gurukul Programme) ལས་འཆར་འདི་བཞིན་ཐོག་མར་ཕྱི་ལོ་ ༡༩༩༤ ལོར་འགོ་བཙུགས་གནང་བ་དང། དེ་སྔ་ལས་འཆར་འདིའི་འོག་མཐོ་སློབ་ཀྱི་སློབ་མ་དྲུག་བདུན་ཙམ་ལས་མཉམ་ཞུགས་གནང་མཁན་མེད་ཀྱང་། འདས་པའི་ལོ་དྲུག་གི་སྔོན་ཙམ་ནས་ལས་འཆར་དེར་མཉམ་ཞུགས་གནང་མཁན་མང་དུ་ཕྱིན་ནས་ལོ་ལྟར་སློབ་མ་བུ་བཅོ་ལྔ་དང་བུ་མོ་བཅོ་ལྔ་བདམས་ནས་རྡ་སར་རྟེན་གཞི་བྱས་པའི་བོད་པའི་ཆོས་དང་རིག་གཞུང་། གོམས་གཤིས་སོགས་ཉམས་མྱོང་གསོག་རྒྱུ་ཙམ་མ་ཟད། བོད་མིའི་སྒྲིག་འཛུགས་ཀྱི་ལས་ཁུངས་མ་ལག་དང་བཅས་པ་ཤེས་རྟོགས་བྱ་རྒྱུ་ཡིན་འདུག་ལོ་ ༢༠༡༤ ཟླ་ ༦ ཚེས་ ༡༣ ཉིན་གྱི་སྔ་དྲོའི་ཆུ་ཚོད་ ༩།༣༠ ནས་ ༡༢།༣༠ ཙམ་བར་བོད་མིའི་སྒྲིག་འཛུགས་ཀྱི་ཕྱི་དྲིལ་དྲུང་ཆེ་བཀྲ་ཤིས་ཕུན་ཚོགས་ལགས་ཀྱིས་ཕྱི་དྲིལ་ལྷག་པ་ཚེ་རིང་དྲན་རྟེན་ཚོགས་ཁང་ནང་རྒྱ་གར་གྱི་རྒྱལ་ས་ལྡི་ལིར་རྟེན་གཞི་བྱས་པའི་༸གོང་ས་༸སྐྱབས་མགོན་ཏཱ་ལའི་བླ་མ་མཆོག་གི་ཀུན་ཕན་བདེ་རྩས་གོ་སྒྲིག་འོག་རྒྱ་གར་རྒྱལ་ཡོངས་མཐོ་སློབ་ཁག་ ༢༩ ནང་སློབ་གཉེར་བྱེད་བཞིན་པའི་སློབ་ཕྲུག་གྲངས་ ༣༣ དང་ཐུག་འཕྲད་ཀྱིས་བོད་མིའི་སྒྲིག་འཛུགས་ཀྱི་སྒྲོམ་གཞི་དང་སྲིད་བྱུས། དེ་བཞིན་ལྷན་ཁང་ཁག་གི་བྱེད་སྒོ་རྣམ་གསུམ་ངོ་སྤྲོད་དང་འབྲེལ་བོད་མིའི་སྒྲིག་འཛུགས་ངོས་ནས་དབུ་མའི་ལམ་གྱི་སྲིད་བྱུས་གཞིར་བཟུང་རྒྱ་ནག་གཞུང་དང་ལྷན་དུ་འབྲེལ་མོལ་བརྒྱུད་བོད་དོན་བདེན་མཐའ་གསལ་ཐབས་ལ་འབད་བརྩོན་ཞུ་བཞིན་ཡོད་སྐོར་སོགས་ཀྱི་གསུང་བཤད་ཟབ་རྒྱས་གནང་བ་མ་ཟད། དྲི་བར་ལན་འདེབས་ཀྱང་གནང་།སློབ་ཕྲུག་དེ་དག་ལ་ཟླ་བ་གཅིག་རིང་བཞུགས་སྒར་རྡ་སའི་ཁུལ་གྱི་བོད་པའི་དགོན་སྡེ་ཁག་ནང་བོད་ཀྱི་ཐུན་མིན་ཆོས་དང་རིག་གཞུང་སོགས་ཀྱི་སྐོར་ངོ་སྤྲོད་དང་སྦྱོང་བརྡར་སྤྲོད་རྒྱུ་ཡིན་འདུག སྤྱིར་སྒུ་རུ་ཀུལ་ (Gurukul Programme) ལས་འཆར་འདི་བཞིན་ཐོག་མར་ཕྱི་ལོ་ ༡༩༩༤ ལོར་འགོ་བཙུགས་གནང་བ་དང། དེ་སྔ་ལས་འཆར་འདིའི་འོག་མཐོ་སློབ་ཀྱི་སློབ་མ་དྲུག་བདུན་ཙམ་ལས་མཉམ་ཞུགས་གནང་མཁན་མེད་ཀྱང་། འདས་པའི་ལོ་དྲུག་གི་སྔོན་ཙམ་ནས་ལས་འཆར་དེར་མཉམ་ཞུགས་གནང་མཁན་མང་དུ་ཕྱིན་ནས་ལོ་ལྟར་སློབ་མ་བུ་བཅོ་ལྔ་དང་བུ་མོ་བཅོ་ལྔ་བདམས་ནས་རྡ་སར་རྟེན་གཞི་བྱས་པའི་བོད་པའི་ཆོས་དང་རིག་གཞུང་། གོམས་གཤིས་སོགས་ཉམས་མྱོང་གསོག་རྒྱུ་ཙམ་མ་ཟད། བོད་མིའི་སྒྲིག་འཛུགས་ཀྱི་ལས་ཁུངས་མ་ལག་དང་བཅས་པ་ཤེས་རྟོགས་བྱ་རྒྱུ་ཡིན་འདུག
\end{document}
